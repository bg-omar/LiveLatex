%! Author = mr
%! Date = 9/3/2025
\newcommand{\canonversion}{\textbf{v0.3.4}} % Semantic versioning: vMAJOR.MINOR.PATCH
\newcommand{\papertitle}{Swirl String Theory (SST) Canon \canonversion}
\newcommand{\paperdoi}{10.5281/zenodo.17014358}

% ==== Swirl String Theory (SST) macros ====
% Context-aware subscript symbol; uses math styles, not \scriptsize
\newcommand{\swirlarrow}{%
    \mathchoice{\mkern-2mu\scriptstyle\boldsymbol{\circlearrowleft}}%
    {\mkern-2mu\scriptstyle\boldsymbol{\circlearrowleft}}%
    {\mkern-2mu\scriptscriptstyle\boldsymbol{\circlearrowleft}}%
    {\mkern-2mu\scriptscriptstyle\boldsymbol{\circlearrowleft}}%
}
\newcommand{\swirlarrowcw}{%
    \mathchoice{\mkern-2mu\scriptstyle\boldsymbol{\circlearrowright}}%
    {\mkern-2mu\scriptstyle\boldsymbol{\circlearrowright}}%
    {\mkern-2mu\scriptscriptstyle\boldsymbol{\circlearrowright}}%
    {\mkern-2mu\scriptscriptstyle\boldsymbol{\circlearrowright}}%
}

% Canonical symbols
\newcommand{\vswirl}{\mathbf{v}_{\swirlarrow}}
\newcommand{\vswirlcw}{\mathbf{v}_{\swirlarrowcw}}
\newcommand{\SwirlClock}{S_(t)^{\swirlarrow}}
\newcommand{\SwirlClockcw}{S_(t)^{\swirlarrowcw}}
\newcommand{\omegas}{\boldsymbol{\omega}_{\swirlarrow}}  % swirl vorticity
\newcommand{\vscore}{v_{\swirlarrow}}                    % shorthand: |v_swirl| at r=r_c
\newcommand{\vnorm}{\lVert \vswirl \rVert}               % swirl speed magnitude
\newcommand{\rhof}{\rho_{\!f}}                           % effective fluid density
\newcommand{\rhoE}{\rho_{\!E}}                           % swirl energy density /c^2? (we define clearly below)
\newcommand{\rhom}{\rho_{\!m}}                           % mass-equivalent density
\newcommand{\rc}{r_c}                                    % string core radius (swirl string radius)
\newcommand{\FmaxEM}{F_{\mathrm{EM}}^{\max}}             % EM-like maximal force scale
\newcommand{\FmaxG}{F_{\mathrm{G}}^{\max}}               % G-like maximal force scale
\newcommand{\Lam}{\Lambda}                               % Swirl Coulomb constant
\newcommand{\Om}{\Omega_{\swirlarrow}}                   % swirl angular frequency profile
\newcommand{\alpg}{\alpha_g}                             % gravitational fine-structure analogue

% Policy: the golden constant is only allowed via hyperbolic functions.
% Never write (1+\sqrt{5})/2; always use \xig=\asinh(1/2), \varphi=e^{\xig}.
\newcommand{\xig}{\operatorname{asinh}\!\left(\tfrac{1}{2}\right)} % base hyperbolic scale  "golden" constant is fundamentally hyperbolic.
\newcommand{\phig}{\exp(\xig)}                                     % golden from hyperbolic
\newcommand{\phialg}{\bigl(1+\sqrt{5}\bigr)/2}                     % algebraic echo (use sparingly)
\newcommand{\xigold}{\tfrac{3}{2}\,\xig}                           % "golden rapidity" scale

% --- Display helpers (optional) ---
\newcommand{\GoldenDeclare}{%
    \textbf{Golden (hyperbolic)}:\ \(\ln\phi=\xig\), hence \(\phi=\phig\).
    \ \emph{(Equivalently, \(\phi=\phialg\); this algebraic form is derivative.)}%
}
% --- Canonical identity (hyperbolic-only proof, algebraic as corollary) ---
\newtheorem{identity}{Identity}


% Preamble
\documentclass{article}

\usepackage{hyperref}
\usepackage{multicol}
\usepackage{calc,pict2e,picture}
\usepackage{textgreek,textcomp,gensymb,stix}
\usepackage{longtable}
\usepackage{amssymb} % For \checkmark


% Packages
\usepackage{amsmath}
\title{Sample \LaTeX}
% Document
\begin{document}
    \maketitle

    \begin{abstract}
        \vswirl
        \vswirlcw
        \SwirlClock
        \SwirlClockcw
        \omegas
        \vscore
        \vnorm
        \rhof
        \rhoE
        \rhom
        \rc
        \FmaxEM
        \FmaxG
        \Lam
        \Om
    \end{abstract}

    \section{Text}
    Text, paragraphs and ligatures goes here.

    \section{Characters}
    Sample characters:
    \$ \& \% \# \_ \{ \} \~{} \^{}

    \section{Math}
    Sample math formulae:
    $f(x) = \int_{-\infty}^\infty \hat f(\xi)\,e^{2 \pi \xi} \, d\xi$

    \section{Multicolumn}
    \begin{multicols}{3}
        Column 1
        Column 2
        Column 3
    \end{multicols}

    \section{Boxes}
    \medbreak\noindent\fbox{\verb|\mbox{|\emph{Sample box}\verb|}|}\smallbreak

    \section{Symbols}
    Sample symbols:
    \noindent \textfractionsolidus \textdiv \texttimes \textminus \textpm \textsurd \textlnot \textasteriskcentered

    \section{Picture}
    Sample picture:
    \setlength{\unitlength}{0.8cm}
    \begin{picture}(6,5)
        \thicklines
        \put(1,0.5){\line(2,1){3}}
        \put(4,2){\line(-2,1){2}}
        \put(2,3){\line(-2,-5){1}}
        \put(0.7,0.3){$A$}
        \put(4.05,1.9){$B$}
        \put(1.7,2.95){$C$}
        \put(3.1,2.5){$a$}
        \put(1.3,1.7){$b$}
        \put(2.5,1.05){$c$}
        \put(0.3,4){$F=\sqrt{s(s-a)(s-b)(s-c)}$}
        \put(3.5,0.4){$\displaystyle s:=\frac{a+b+c}{2}$}
    \end{picture}

testo
\begin{equation}
E = mc^2
\end{equation}
no equation number
\[ \frac{1}{2}mv^2\]


inline equation $a^2 + b^2 = c^2$ type with dollar signs
and also with \(a^2\) like this \(e^{i\pi} + 1 = 0\)

But does it support \textbf{bold} and \textit{italic} text?

yes it does, and also \underline{underline}.
\section{New Section} This is a new section
\subsection{New Subsection} This is a new subsection
\subsubsection{New Subsubsection} This is a new subsubsection
\paragraph{New Paragraph} This is a new paragraph

\subparagraph{New Subparagraph} This is a new subparagraph

% Example table
\begin{table}[h!]
  \begin{center}
    \caption{Your first table.}
    \label{tab:table1}
    \begin{tabular}{l|c|r}
      \textbf{Value 1} & \textbf{Value 2} & \textbf{Value 3}\\
      $\alpha$ & $\beta$ & $\gamma$ \\
      \hline
      1 & 1110.1 & a\\
      2 & 10.1 & b\\
      3 & 23.113231 & c\\
    \end{tabular}
  \end{center}
\end{table}

\end{document}